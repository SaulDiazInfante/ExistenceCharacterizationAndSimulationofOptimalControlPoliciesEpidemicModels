Here the main reference is the book of Lenhart 
\cite{Lenhart2002}.
\paragraph{Introduction.}
  Since the first days of humanity, infectious diseases sculpt civilizations. 
For example, by the end of the Middle Ages, smallpox considerably cut down the 
population in centers of Europe and Asia---3 of  each ten dies by smallpox, 
perhaps that gives its alias, "speckled monster."  Although experts understand
the mechanism of transmission of this  "monster" until the early 20  century, it
represents the first documented disease
\citep[][]{bernoulli1760essai, bradley1971smallpox, Foppa2017}
against which a specific control intervention was available: the inoculation.
This process relies on put material from smallpox sores healthy people. Usually
scratching material over the armor or inhaling it through the nose. People
usually develop the symptoms associated with smallpox---fever and a rash.
However, the death rate due to inoculation is considerably lower than natural
smallpox.

  Then Bernoulli naturally sets a question like this: What happens
if everybody were inoculated?. Here, we address the question: How to 
inoculated in optimal way?.

\paragraph{What means a optimal control policy in this context?}

\paragraph{The cost function and its interpretation in epidemic control models.}

\paragraph{Comments about recent bibliography.}
    The approach in this work relies on Pontryagin’s Maximum Principle
\cite{} and follows the same methodology of Lenharts.
In our opinion Lenhart's work makes accessible a optimal control device to
describe common epidemic control intervention like vaccination, treatment, 
quarantine, isolation among others. Our intention in this work is illustrate 
the mentioned strategies throughout recent literature. Likewise we present 
important regarding goals and issues that appear with the relating theory and 
numerical approximations.
\paragraph{Usual control policies.}
