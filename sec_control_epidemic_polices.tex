  Here we present  several control models that we consider as good 
examples. Before talking about this good examples, we give the core of optimal 
policy modeling, see for example the survey of \citet{Wickwire1977}, for more
details.

  First, we require a model to describe the spreading of an uncontrolled
disease, and whose transitions generate a cost. Then, we add a continuous
control action to modify the changes from one state to another but in such way
that the mentioned cost is optimized. A rule that prescribes which control
operation to use at each time, is a control policy. A control policy which
applies only information from the current state of the controlled system to
prescribe control actions is a \emph{closed-loop} or \emph{feedback} control.
If the current state is not observable, or the control function only depends 
on the time we have an \emph{open-loop} policy: the sort of policies that we 
consider in this work.

  Here, we consider control policies that affect the bounded rates at which
population moves from one class (e.g., infected) to another (e.g., recovered).
In all these problems, the control function appears linearly in the relevant
dynamic. Next, we specify a cost functional which assigns the total cost of the
control policy implementation. Then the problem is to determine a policy that
optimizes the regarding cost strategy.

  Now we present the mentioned good examples. In what follows $X$ denotes a 
vector with all concerning populations, for example, according to SIR model 
\eqref{eqn:sir_model},  $X=(S, I, R)^\top$ .
