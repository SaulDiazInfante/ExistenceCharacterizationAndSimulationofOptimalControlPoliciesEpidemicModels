%Here the main reference is the book of Lenhart \cite{Lenhart2002}.

\paragraph{Introduction.} 
Here we presented several control models that we consider as good 
examples. Before talking about this good examples, we give the core of optimal 
policy modeling.

First, we require a model to describe the spreading of an uncontrolled disease,
and whose transitions generate a cost. Then, we add a continuous control action
to modify the changes from one state to another but such that we optimize the
mentioned cost. A rule which prescribes which control operation to use at each
time is a control policy. A control policy which applies only information from
the current state of the controlled system to prescribe control actions is a
closed-loop or feedback control. If the current state is not observable, or the
control function only depends on the time we have an open-loop policy: the sort
of policies that we consider in this work.

  Here, we consider control policies that affect the bounded rates at which
population moves from one class (e.g., infected) to another (e.g., recovered).
In all these problems, the control function appears linearly in the relevant
dynamic. Next, we specify a cost functional which assigns the total cost of the
control policy implementation. Then the problem is to determine a policy that
optimizes the regarding cost strategy.

  A common practice to solve this kind of problems follows the next steps:
  \begin{enumerate}[(a)]
    \item
      Prove that exist an optimal policy.
    \item 
      Find and use necessary condition for the optimality of a policiy.
      Following sections provide a technique to transform a given optimization
      problem into solve a ordinary differential equation with boundary values.
    \item 
      From the necessary conditions, determine important qualities of the 
      optimal policies, for example, the boundedness of controls.
    \item 
      Usually this kind of problems are non linear, then find solution is 
      extremely difficult. Therefore, choose a convenient numerical scheme is
      very important. In this work we implements the forward-backward-sweep 
      method.
  \end{enumerate}
\paragraph{Comments about recent bibliography.}

\paragraph{Usual control policies.}
  \subparagraph{Vaccination}
  \subparagraph{Case finding and case holding}
  \subparagraph{Isolation and quarantine}
  \subparagraph{Culling}