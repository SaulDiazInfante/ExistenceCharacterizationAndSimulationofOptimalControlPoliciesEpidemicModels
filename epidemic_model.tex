Usually, public health organizations consider vaccination as a
primarily preventive action against infectious diseases but it incurs a cost.%is plausible only when represents a benefit. 
Due to the limited resources associated with vaccination programs, it is imperative
to optimize the use of available resources. Using optimal control theory, we
formulate a vaccination schedule. The goal is to minimize the number of infected
persons and the cost of vaccine during a fixed time, for this example, we 
optimize the functional
$$
  \int_{0}^{T}
    A I(t) + u^2(t) .
$$
Here $u$ is the vaccination control and denotes the fraction of
susceptible individuals to vaccinate per unit of time. Since managing infected 
population imply resource consumption, $A$ represents the cost per individual.
We also need a spread dynamics. So, let $S(t)$, $E(t)$, $I(t)$, $R(t)$
respectively be the number of susceptible, exposed, infectious, and recovered
(immune) individuals at time $t$. Since vaccination of
the entire susceptible population is impractical, the model considers 
$0 \leq u(t) \leq 0.9$. Then the whole population $N$ is given by 
$N(t) = S(t) + E(t) + I(t) + R (t)$, and obeys
$
  \dot{N}(t) =
    (b - d)N(t) - aI(t).
$
Since $b$ is the recruitment rate and $d$ natural death, the term $b-d$ denotes 
the growth of the entire population. Then, the optimal control problem reads
%
%
\begin{equation} \label{eqn:epidemics_lenhart}
  \begin{aligned}
    \min_{u} & \int_{0}^{T} AI(t) + u^{2}(t) dt,
    \\
    \text{subject to}
    \\
      \dot{S}(t) &=
          bN(t) - dS(t) - cS(t)I(t) - u(t)S(t), \quad S(0) = S_0 \geq 0,   \\
      \dot{E}(t) &=
          cS(t)I(t) - (e + d)E(t), \quad E(0) = E_0 \geq 0,    \\
      \dot{I}(t) &=
          eE(t) - (g + a +d)I(t), \quad I(0) = I_0 \geq 0,     \\
      \dot{R}(t) &=
          gI(t) -dR(t) + u(t)S(t), \quad R(0) = R_0 \geq 0,    \\
        \dot{N}(t) &=
          (b - d)N(t) - aI(t), \quad N(0)= S_0 + E_0 + I_0 + R_0, 
  \end{aligned}
\end{equation}
see \Cref{tbl:epidemics_lenhart_des} for a description of the parameters.

\begin{table}[H]
  \begin{center}
    \begin{tabular}{rl}
      \toprule
        & \multicolumn{1}{c}{\textbf{Parameter Description}} 
        \\
      \midrule
        $b$
          & Recruitment rate
        \\
        $a$, $d$ 
          & Disease and natural death rates
        \\
        $c$
          & Incidence of disease
        \\
        $e$
          & Rate at which the exposed 
          \\
          & individuals become infectious
        \\
        $g$
          & Recovering rate
        \\
        $A$
          & Vaccination cost
        \\
        $T$
          & Final time
        \\
      \bottomrule
    \end{tabular}
    \caption{Parameters and simulation values of the epidemic model
      \eqref{eqn:epidemics_lenhart}.}
    \label{tbl:epidemics_lenhart_des}
  \end{center}
\end{table}
%