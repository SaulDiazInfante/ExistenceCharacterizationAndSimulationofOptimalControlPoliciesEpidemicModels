
%\section{Existence and characterization of optimal policies}

The non controlled epidemic models described above are of the form 
    \begin{eqnarray*}
\dot{X} & = & AX +  
    \begin{bmatrix}
    X^\top B^{(1)}\\
    \vdots \\
    X^\top B^{(n)}
  \end{bmatrix}X + k  \\
  & = & \left(A + [X^\top \cdots X^\top]     \begin{bmatrix}
    B^{(1)}\\
    \vdots \\
    B^{(n)}
  \end{bmatrix} \right) X + k
    \end{eqnarray*}
where the matrix $A$ represents the linear part of the system, each matrix $B^{(j)}$, $j=1,\ldots,n$, gives the {\it interaction} part as a quadratic form, and $k$ is a constant vector. Thus the $j$-th row of the system takes the form 
    \[ \dot{X}_j = r_j(A)X + X^\top B^{(j)}X + k_j.\]

In this section we consider control policies in both the linear and the interaction parts of the latter system. This family of control systems with a corresponding  cost functional include the above epidemic models. 

For such a family of optimal control problems we state and prove three main results. First, given any control policy, we establish the existence and uniqueness of the associated state path. Second, the existence of an optimal control policy is proved. Finally, by means of the Maximum Principle, sufficient conditions on a control policy and the corresponding state path are also given. Our proofs are based on general and well-known results in optimal control theory which, for completeness, are stated in the Appendix at the end of the paper. 

\subsection{A family of control systems}


Let $\mathbf{X}\s\R^n$ and $\mathbf{U}\s \R^m$ be nonempty and compact sets. The sets $\mathbf{X}$ and $\mathbf{U}$ are 
respectively called the {\it state space} and the {\it control space}. The vectors in $X$ have non-negative entries, in particular we assume that $0\in X$. The control set $\mathbf{U}$ is convex. We consider the following control system, for $j=1,\ldots,n$, 
\begin{equation}\label{FamilyControlSystem}
    \dot{X}_j  =  [r_j(A) + u^\top C^{(j)}]X +
    [X^\top \cdots X^\top]     \begin{bmatrix}
    r_1(B^{(j)}) + u^\top D^{(j1)}\\
    \vdots \\
    r_n(B^{(j)}) + u^\top D^{(jn)}
  \end{bmatrix}  X + k_j
\end{equation}
where $A\in\R^{n\times n}$, $B^{(j)}\in\R^{n\times n}$, $C^{(j)}\in\R^{m\times n}$, and $ D^{(jl)}\in\R^{m\times n}$ for $l=1,\ldots,n$.


%In order to guarantee the existence of a solution $x$ to \eqref{CoDiffEq}, 
%we need the following.




The proof of the following theorem slightly differs from that given in Yong \cite[Sect. 2.1]{Yong2015} since we consider a weighted norm. This improvement allows us to give a global solution instead of a local one. 

\begin{theorem}\label{ExAdmisPair} 
For each measurable function 
	$u:[0,T]\to \mathbf{U}$ and each initial condition $x_0\in X$, there exists a unique absolutely continuous function $X_u:[0,T]\to \R^n$ that satisfies the the system 
	\eqref{FamilyControlSystem} almost everywhere. %The solution $x_u$ satisfies	\begin{equation}\label{ineqSol} 
	%	|x_u(t)|\leq e^{Lt}(1+|x_0|) -1,\qquad t\in [0,T].	\end{equation}
\end{theorem}
\begin{proof} Let $u:[0,T]\to U$ be a measurable function. The control system \eqref{FamilyControlSystem} can be written as
\[\dot{X}(t)=f(X(t),u(t)), \quad X(0)=x_0,\quad 0\leq t\leq T,\]
where $f:\mathbf{X}\times \mathbf{U}\to \R^n$. Since $f$ is of class $\mathcal{C}^1$ on the compact set $\mathbf{X}\times \mathbf{U}$, there exists a constant $L>0$ such that
\begin{eqnarray}
  \|f(x,u)-f(x_1,u)\| & \leq & L\|x-x_1\|\label{LipfInx}\\
  \|f(0,u)\| & \leq & L\label{fBound}
\end{eqnarray}
for every $x,x_1\in \mathbf{X}$ and $u\in \mathbf{U}$.

 Consider the linear space 
    \[\mathbb{X}=\{X:[0,T]\to \R^n\mid X \mbox{ is continuous}\}\] 
with the norm
    \[ \|X\|_w:=\sup_{t\in[0,T]} \frac{\|X(t)\|}{w(t)}, \]
where $w(t):=e^{Lt}$ for each $t\in [0,T]$. It can be shown, with slight modifications in \cite[Section 2.1]{Teschl}, that the pair $(\mathbb{X},\|\cdot\|_w)$ is a Banach space. Define the operator $K:\mathbb{X}\to \mathbb{X}$ by 
    \[ K[X](t):=x_0 + \int_0^t f(X(s),u(s))ds.\]
By \eqref{LipfInx} and \eqref{fBound}, any $(x,u)$ satisfies 
  \begin{equation}
      \|f(x,u)\| \leq  L(1+\|x\|),
  \end{equation}
thus $f(X(\cdot),u(\cdot))$ is Lebesgue integrable and $K[X]$ is absolutely continuous. We claim that $K$ is a contraction with contraction constant $1-e^{-LT}$. Indeed,
    \begin{eqnarray*}
    \| K[X] - K[Y] \|_w & = & \sup_{t\in[0,T]} \frac{|\int_0^t [f(X(s),u(s)) -f(Y(s),u(s))]ds|}{w(t)}\\
        & \leq &   \sup_{t\in[0,T]} \frac{L\int_0^t w(s)[w(s)]^{-1}|X(s) -Y(s)|ds}{w(t)}\\
        &\leq &  L\|X-Y\|_w \sup_{t\in[0,T]} \frac{\int_0^t w(s)ds}{w(t)}\\
        & = &  L\|X-Y\|_w \sup_{t\in[0,T]}\frac{[e^{Lt}-1]/L}{e^{Lt}}\\
        & = &  (1-e^{-LT})\|X-Y\|_w. 
    \end{eqnarray*}
Then by Banach's fixed point theorem \cite[Theorem 2.1]{Teschl}, there exists a unique $X\in \mathbb{X}$ satisfying 
    \[ X(t)=x_0 + \int_0^t f(X(s),u(s))ds.\]
Therefore \eqref{FamilyControlSystem} holds almost everywhere \cite[Corollary 5.4.1]{Loeb2016}.
\end{proof}


\subsection{Existence of optimal policies}




Consider the  {\it cost functional} of an admissible control $u$, given the initial state $x_0$, 
        \begin{equation}\label{CostFunctional} V(u,x_0) := \int_0^Tg(X(t),u(t))\,dt,\end{equation}
where $g: \mathbf{X}\times \mathbf{U}\to \R$ is continuous. The {\it optimal control problem} (OCP) consists of finding an admissible control $u^\ast$ such that
\[ V(u^\ast,x_0)=\inf\{ V(u,x_0)\mid u\in \mathbb{U}_B \}.\]
If there exists such a control $u^\ast$, then it is called an {\it optimal policy} or {\it optimal control}. The pair $(u^\ast,X^\ast)$, where $X^\ast$ is given by Theorem \ref{ExAdmisPair}, is called an {\it optimal pair}.


\begin{theorem} Suppose the function $g$ is continuous, and, for each $x$,  the function $g(x,\cdot)$ is convex, i.e.,
        \[  \alpha g(x,u_1) +(1-\alpha) g(x,u_2) \geq g(x,\alpha u_1+(1-\alpha)u_2) \quad \forall u_1,u_2\in\mathbf{U},\  \alpha\in [0,1]. \]
Then there exists an optimal pair that minimizes \eqref{CostFunctional} subject to \eqref{FamilyControlSystem}. 
\end{theorem}
\begin{proof} Let us write the control system \eqref{FamilyControlSystem} as $\dot{X}=f(X,u)$. By Filippov's Theorem \ref{FilipovThm}, it is enough to show that each set   \[ 
 \{ (z, y)\in \mathbb{R}\times \mathbb{R}^n\mid  
    z \geq g(x,u), \  y=f(x,u), \ u\in \mathbf{U}\},\qquad x\in X,
     \]  
is convex. Fix $x\in X$. Let  $z_1,z_2\in \R$ and  $y_1,y_2\in \R^n$ such that
      \begin{equation}\label{rj>g}
          z_j\geq g(x,u_j),\quad   j=1,2, 
      \end{equation}
and 
    \begin{equation}\label{yj=f}
  y_j=f(x,u_j),\quad j=1,2, 
      \end{equation}
for some $u_1,u_2\in U$. We need to show that, for any $\alpha\in[0,1]$, there exists $u'\in U$ such that
    \begin{equation}\label{ccr>g}
    \alpha z_1 + (1-\alpha)z_2 \geq g(x,u') \end{equation}
and 
    \begin{equation}\label{ccy=f}
 \alpha y_1 + (1-\alpha)y_2 = f(x,u'). \end{equation}
Let $u':=\alpha u_1 + (1-\alpha)u_2$. Then \eqref{ccr>g} follows from \eqref{rj>g} and the convexity of $g(x,\cdot)$. On the other hand, \eqref{ccy=f} holds because $f(x,\cdot)$ is affine, i.e.,  
  \[ f(x,\alpha u_1+(1-\alpha)u_2) = \alpha f(x,u_1) +(1-\alpha) f(x,u_2). \]
\end{proof}


\subsection{Sufficient conditions for optimality}

Consider the Hamiltonian $\mathcal{H}:\mathbf{X}\times\mathbf{U}\times\R^n\to \R$, defined as 
    \[\mathcal{H}(x,u,\lambda):= g(x,u) + \lambda^\top f(x,u),\]
and
\[ \mathcal{H}^\ast(x,\lambda)  :=  \inf_{u\in\mathbf{U}}\mathcal{H}(x,u,\lambda),\]
where $g$ determines the cost functional \eqref{CostFunctional} and $f$ is given by the right-hand side of the control system \eqref{FamilyControlSystem}.

 The function $\lambda:[0,T]\to \R^n$ and the admisible pair $(u,X)$ are said to satisfy the necessary conditions of the {\it Maximum Principle} (MP) if they satisfy the
 {\it adjoint equation}
\begin{equation}\label{AdjEqu}
         \dot{\lambda}(t) = -\mathcal{H}_x(X(t),u(t),\lambda(t))^\top, \quad \lambda(T)=0,
    \end{equation}
    and the {\it optimality condition}
 \begin{equation}\label{OptCond}
       \mathcal{H}^\ast(X(t),\lambda(t)) =
\mathcal{H}(X(t),u(t),\lambda(t)).
   \end{equation}





\begin{definition}\label{PiecewiseCont}\rm The function $w$ from $[0,T]$ to some Euclidean space is {\it piecewise continuous} if
\begin{enumerate}[(a)]
    \item $w$ is continuous on $[0,T]$ except at a finite number of points, and 
    \item if $w$ is discontinuous at $t$, then  
        \[\lim_{s\to t^-}w(s) \mbox{ and } \lim_{s\to t^+}w(s)\]
        are finite. 
\end{enumerate}
\end{definition}



\begin{theorem} Let $\lambda:[0,T]\to \R^n$  be a continuous function and let $(u^\ast,X^\ast)$ be an admissible pair such that 
\begin{enumerate}[\rm (a)]
    \item $u^\ast$ is piecewise continuous,
    \item $\dot{X}^\ast$ exists and is piecewise continuous,
    \item $(\lambda,u^\ast,X^\ast)$ satisfies the optimality condition \eqref{OptCond}, and,
    \item except at the points of discontinuity of $u^\ast$, the adjoint equation \eqref{AdjEqu} holds.
\end{enumerate}
If, for each $t$, the function $\mathcal{H}^\ast(\cdot,\lambda(t))$ is convex on $\mathbf{X}$, then $(u^\ast,X^\ast)$ is an optimal pair.
\end{theorem}

\begin{proof} The conclusion follows from Theorem \ref{SufficientCond}, whenever Assumptions \ref{ContinuityH} and \ref{piecewise} hold. 
If $u^\ast$ is piecewise continuous, then $X^\ast$ is absolutely continuous, by Theorem \ref{ExAdmisPair}, and so continuous. Thus Assumption \ref{piecewise} holds. Assumption \ref{ContinuityH} also holds since $\mathcal{H}$ and $\mathcal{H}_x$ are clearly continuous. This completes the proof.
\end{proof}