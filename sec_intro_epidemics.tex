  Infectious diseases have struck civilizations in different periods of human 
history.  HIV AIDS, Spanish influenza, and Black Death are the most devastating
pandemics, they have killed more than 100 million people. Therefore,
understanding the mechanism of spread and control of diseases of this kind is
essential. In this line, the SIR structure is a convenient option to model its
spreading.

  The SIR model is a compartmental structure. Primarily, the model consists of 
three compartments: susceptible $S$, infected $I$, and recovered $R$, %that 
 and transitions functions between compartments.

  Practically, all the existing epidemic models are variants of this structure. 
The variants emerge to describe particular characteristics of a disease, 
mechanism of transmission, population dynamics, among others. 
To fix ideas, consider the classic model of 
\citet{Kermac}
\begin{equation}
  \label{eqn:sir_model}
  \begin{aligned}
    \frac{dS}{dt} & = - \kappa SI
      \\
    \frac{dI}{dt} & = \kappa SI - \lambda I
      \\
    \frac{dR}{dt} & = \lambda I,
  \end{aligned}
\end{equation}
here the transition from the susceptible $S$ to the infected class $I$ 
occur with constant rate $\kappa$, and from  the infected class $I$ to the 
recovered happens with rate $\lambda$.

In next sections, we provide the main ideas to modify this 
basic structure with control policies.