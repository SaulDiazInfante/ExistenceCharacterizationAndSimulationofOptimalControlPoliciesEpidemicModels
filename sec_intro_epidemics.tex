  Infectious diseases sculpt civilizations. The three most devastating 
pandemics in human history: the Black Death, Spanish influenza, and HIV/AIDS
have been killed more than 100 million people. Therefore, understand the
mechanism of spreed and control of diseases of this kind is wide important.

  In this line, the SIR structure is a first option to model spreading. The SIR 
model is a compartmental structure. Essentially, the model consists of three 
compartments: $S$, $I$, $R$, that respectively represent the number of 
susceptible infected and recovered, and certain transitions functions between 
classes.

  Practically all the existing epidemic models are variants of this structure. 
The variants emerge to describe particular characteristics of a disease, 
mechanism of transmission, population dynamics, among others. 
To fix ideas, consider a very  popular epidemic models reported by 
\citet{Kermac}
\begin{equation}
  \begin{aligned}
    \frac{dS}{dt} & = - \kappa SI
      \\
    \frac{dI}{dt} & = \kappa SI - \lambda I
      \\
    \frac{dR}{dt} & = \lambda I,
  \end{aligned}
\end{equation}
here the transition from the susceptible $S$ to the infected class $I$ 
occur with constant rate $\kappa$, and from  the infected class $I$ to the 
recovered happens with rate $\lambda$.

In next sections, we provide the main ideas to control modifications of this 
base structure with optimal policies.