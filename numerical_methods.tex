\subsection{Popular methods}
Since we can transform the problem of optimal control into a two-point 
boundary ODE problem, the methods designed for this sort of problem are 
applicable see \cite{Keller1976, Ascher1987, Stoer2013} for classic references. 
In this line \cite{Caetano2001,Yan2008} use multiple shooting methods to 
solve the resulting extended two-value boundary ode.
\paragraph{Multiple shooting method}
Roughly speaking, the multiple shooting method follows the next algorithm.
Consider the controlled dynamics and corresponding adjoint equations given by
\begin{equation}
  \label{eqn:extended_tpvbp}
  \begin{aligned}
    \dot{x}(t) &= 
    f(x(t), u(t)), \qquad x(0)=x_0 \\
    \dot{\lambda}(t) &=
    -\mathcal{H}_x(t,x(t),u(t),\lambda_0,\lambda(t))^\top, \quad 
    \lambda(T)=0.
  \end{aligned}
\end{equation}
Given  partition of the interval $[0, T]$ with uniform step $h$,
$$
\tau_h^n:= \{t_k = kh, \ k=0,\dots n\},
$$
the multi shooting method follows the \Cref{alg:multishooting}.
\begin{algorithm}
  \caption{Multi shooting method } \label{alg:multishooting}
  \begin{flushleft}
    \hspace*{\algorithmicindent} \textbf{Input:} 
    $t_0, T, x_0, h, \text{tol}, \lambda_{f}, n_{max}$ \\
    \hspace*{\algorithmicindent} \textbf{Output:} 
    $x^*, u^*, \lambda$
  \end{flushleft}
  \begin{algorithmic}
    \Procedure{Multi  shooting}{$g,\lambda_{\text{function}}, 
      u, x_0, 
      \lambda_f, h, n_{max}$} 
      \While{$ \epsilon > \text{tol}  $}
        \State 
         Choose $y_i := [x(t_i ), \lambda(t_i )], \quad i = 1,\dots, n$.
        \\
        \State 
          Integrate \eqref{eqn:extended_tpvbp} for each sub-interval 
          $[t_i , t_{i+1})$ using $y_i$ as the initial conditions 
          \\
          \hspace{.98cm}
          and obtain 
          $y(t_{i−1}) = [x(t_{i−1}), \lambda(t_{i−1})]$, 
          $i=2, \dots, n$.
        \\
        \State 
          $\mathcal{Y} \gets [y_i - y(t_i)]$, $i=0, \dots, n$
        \State
          Actualize initial condition $y_i$ for next iteration
          using for a example a
          \\
          \hspace{.98cm}
           Newton's method.
        \State
          $\epsilon \gets |\mathcal{Y}|$  
      \EndWhile
     \EndProcedure
  \end{algorithmic}
\end{algorithm}

\paragraph{The Forward-Backward-Sweep}
  However, the forward-backward-sweep method is the most popular method in 
  works about optimal control epidemic models. Perhaps for its easy 
  implementation and acceptable convergence, this method is the most used
  on this kind of applications. In fact, all simulations presented in this work
  runs with this scheme. \citet{hackbusch1978numerical} propose this numerical
  scheme to solve a class of optimal problems that encloses our applications.
  \citet{lenhart2007optimal} provides MATLAB code for many of its regarding
  work in biological models.
\cite*{Pesch1989} , [Saúl]
\paragraph{Genetic Algorithms}
[Frank]
\begin{algorithm}
  \caption{Forward Backward Sweep } \label{alg:forward_backward_sweep}
  \begin{flushleft}
    \hspace*{\algorithmicindent} \textbf{Input:} 
    $t_0, t_f, x_0,h, \text{tol}, \lambda_{f}$ \\
    \hspace*{\algorithmicindent} \textbf{Output:} 
    $x^*, u^*, \lambda$
  \end{flushleft}
  \begin{algorithmic}
    \Procedure{Forward backward sweep}{$g,\lambda_{\text{function}}, 
      u, x_0, 
      \lambda_f, h, n_{max}$} 
    \While{$ \epsilon > \text{tol}  $}
      \State $u_{\text{old}} \gets u$ 
      \State $x_{\text{old}} \gets x$ 
      \State $ x \gets$
      \State $\lambda_{\text{old}} \gets \lambda $
      \Call{runge\_kutta\_forward}{$g, u, x_0, h$}
      \State $\lambda \gets$ 
        \Call{runge\_kutta\_backward}{%
       $\lambda_{\text{function}}, x, \lambda_f, h$}
      \State $u_1 \gets$ 
        \Call{optimality\_condition}{$u, x, \lambda$}
    %
      \State 
        $u \gets \alpha u_1 + (1-\alpha)u_{old}, 
        \qquad \alpha \in [0, 1]$
        \Comment{convex combination}
      \State 
      $\epsilon_u \gets \displaystyle 
        \frac{||u - u_{\text{old}}||}{||u||}$
      \State 
      $\epsilon_x \gets \displaystyle 
        \frac{||x - x_{\text{old}}||}{||x||}$
      \Comment{relative error}
      \State 
        $\epsilon_{\lambda} \gets \displaystyle 
        \frac{||\lambda - \lambda_{\text{old}}||}{||\lambda||}$
      \State 
        $\epsilon \gets 
        \max{ 
          \{ \epsilon_u, \epsilon_x, \epsilon_{\lambda} \}
       }$
    \EndWhile\label{}
      \State \textbf{return} $ x^*, u^*, \lambda$
        \Comment{Optimal pair}
    \EndProcedure
  \end{algorithmic}
\end{algorithm}

