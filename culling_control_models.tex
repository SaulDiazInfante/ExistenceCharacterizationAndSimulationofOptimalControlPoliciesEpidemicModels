Pathogens that are transmitted between wildlife, livestock and humans present
major challenges for the protection of human and animal health: the economic
sustainability of agriculture and the conservation of wildlife. Mycobacterium
bovis, the aetiological agent of bovine tuberculosis (TB), is one such pathogen.
For example, according with \citet{Donnelly2003} the incidence of TB in cattle 
has increased substantially in parts of Great Britain in the past two decades, 
adversely affecting the livelihoods of cattle farmers and potentially 
increasing the risks of human exposure. The control of bovine TB in Great 
Britain is complicated by the involvement of wildlife, particularly badgers 
which appear to sustain endemic infection and can transmit TB to cattle. 
Between \num{1975} and \num{1997} over \num{20000} badgers were
culled as part of British TB control policy, generating conflict between
conservation and farming interest groups.

\subsection*{Badger bovine tuberculosis}
\citet{Bolzoni2014} reports a controlled model to describe a outbreak of badger 
bovine tuberculosis. The regarding uncontrolled model reads

\begin{equation}\label{eqn:culling}
	\begin{aligned}
  \min_{u(t)\in \mathcal{U}}
    &
    \int_0^T
      I(t) + P [u(t)]^{\theta}, \quad \theta \in \{1,2\},
      \quad P = B/A
  \\ \textrm{subject to:} &
  \\
    &\dfrac{dS}{dt} =
			r S 
			\left (
				1 - \dfrac{S+I}{K}
			\right)
			 - \beta SI - u(t) S
		\\
		&\dfrac{dI}{dt} =
			\beta SI - (\alpha + \mu + u(t)) I.
	\end{aligned}
\end{equation}
%
Here the susceptible follow a logistic growth with net grow rate
$r = \nu - \mu$ and carrying capacity $K$. According with the approach, 
of \citet{VandenDriessche2017}, the regarding reproductive number results
$$
  R_0 = \frac{\beta K}{\alpha + \mu}.
$$ 
Our intention with this model is illustrate the difference between quadratic 
and bang-bang controls. As will see, according with model \eqref{eqn:culling}.
The resulting controlled dynamics with this kind of controls is similar in some 
cases, but is unclear which is better.
\begin{table}
  \begin{center}
    \begin{tabular}{@{}rl@{}}
        \toprule
      \multicolumn{2}{c}{\bf{Description}}
      \\
      \midrule
      $\nu$
        &
          Natural fertility rate
      \\
      $\mu$
        & Natural mortality rate
      \\
      $K$
        & Carrying capacity
      \\
      $\alpha$
        & Disease-induced mortality rate
      \\
      $R_0$
        & Basic reproductive number
      \\
      $\beta$
        & Transmission coefficient
      \\
      $P$
        & Relative cost per unit culling effort \\
        & over the cost of a single infection
      \\
      \bottomrule
    \end{tabular}
  \end{center}
  \caption{Parameter description of the control model \eqref{eqn:culling}.}
  \label{tbl:culling_parameter_des}
\end{table}