  By the end of the Middle Ages, smallpox cut down the 
population in centers of Europe and Asia\textemdash three of each ten dies by 
smallpox\textemdash perhaps that gives its alias, ``speckled monster''.  
Although experts understand the mechanism of transmission of this  ``monster''
until the early 20  century, it represents the first documented disease
\citep[][]{bernoulli1760essai, bradley1971smallpox, Foppa2017} against which a
specific control intervention was available: the inoculation. This process
relies on putting material from smallpox sores to healthy people, usually scratching
material over the armor or inhaling it through the nose. People develop the
symptoms associated with smallpox---fever and a rash. However, the death rate
due to inoculation is considerably lower than natural smallpox.

  Then Bernoulli naturally sets a question like this: What happens if everybody
were inoculated? Here, we address the question: How to inoculate in an optimal
way? Throughout the following lines, we try to answer and illustrate the
implications of this question. 

  Optimal control theory is a way to deal with the above question.  In the
fifties, Pontryagin and Bellman propose generalizations of the calculus of
variations of broader applicability:  the Maximum Principle and the method of
Dynamic Programming, respectively. Now, these results sustain application in the
biological sciences and, in particular, to the optimal control
of infectious diseases, see  \cite{Yu2018,Lahrouz2018,Jang2018,Cai2017b}
for recent literature.

  Our approach in this work relies on Pontryagin's Maximum
Principle \cite{pontryagin1962} and follows the same methodology of
\citet{lenhart2007optimal}. Lenhart's work makes an accessible optimal control
device to describe common epidemic interventions like vaccination, treatment,
quarantine, isolation among others. Our intention in this work is illustrating
the mentioned strategies throughout recent literature and state results for
existence and characterization of optimal controls for a particular family of epidemic models.
Likewise we present some goals and issues that appear within the
 theory and numerical approximations. 
 
 The paper is organized as follows. We start in Section 2 by introducing a rather simple but seminal dynamical system from which most of the epidemic models are derived. In Section 3 we describe four epidemic models as well as some control policies. In Section 4 we provide the main theoretical results for a family of optimal control problems (OCPs); such a family includes the models in Section 3. Our proofs are based on well known results---stated, for completeness, in the Appendix of Section 8---from optimal control theory. Some numerical methods to solve OCPs are given in Section 5, in particular, we provide 
the python implementation code of the forward-backward-sweep method in repository \cite{python_repo}. The reader is free to comment, use, improve or whatever he wants about this repository. In Section 6 we run several numerical experiments, based on \cite{python_repo}, for the models described in Section 3. We conclude with some remarks in Section 7.
 
 






















