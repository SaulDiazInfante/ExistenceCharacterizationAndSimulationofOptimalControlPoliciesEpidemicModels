  By the end of the Middle Ages, smallpox cut down the 
population in centers of Europe and Asia\textemdash three of each ten dies by 
smallpox\textemdash perhaps that gives its alias, "speckled monster."  
Although experts understand the mechanism of transmission of this  "monster"
until the early 20  century, it represents the first documented disease
\citep[][]{bernoulli1760essai, bradley1971smallpox, Foppa2017} against which a
specific control intervention was available: the inoculation. This process
relies on put material from smallpox sores to healthy people. Usually scratching
material over the armor or inhaling it through the nose. People develop the
symptoms associated with smallpox---fever and a rash. However, the death rate
due to inoculation is considerably lower than natural smallpox.

  Then Bernoulli naturally sets a question like this: What happens
if everybody were inoculated? Here, we address the question: How to 
inoculated in optimal way? Throughout the following lines we answer and 
illustrate the implications of this question.
The approach in this work relies on Pontryagin's Maximum Principle
\cite{Boltyanski1960} and follows the same methodology of Suzanne Lenhart.
In our opinion, Lenhart's work makes an accessible optimal control device to
describe common epidemic interventions like vaccination, treatment, 
quarantine, isolation among others. Our intention in this work is illustrate 
the mentioned strategies throughout recent literature. Likewise we present 
important regarding goals and issues that appear with the relating theory and 
numerical approximations. We start by fixing our notation and enunciate the core
of the theory.
