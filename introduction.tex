  By the end of the Middle Ages, smallpox cut down the 
population in centers of Europe and Asia\textemdash three of each ten dies by 
smallpox\textemdash perhaps that gives its alias, "speckled monster."  
Although experts understand the mechanism of transmission of this  "monster"
until the early 20  century, it represents the first documented disease
\citep[][]{bernoulli1760essai, bradley1971smallpox, Foppa2017} against which a
specific control intervention was available: the inoculation. This process
relies on put material from smallpox sores to healthy people. Usually scratching
material over the armor or inhaling it through the nose. People develop the
symptoms associated with smallpox---fever and a rash. However, the death rate
due to inoculation is considerably lower than natural smallpox.

  Then Bernoulli naturally sets a question like this: What happens if everybody
were inoculated? Here, we address the question: How to inoculated in optimal
way? Throughout the following lines we answer and illustrate the implications
of this question. 

  Optimal control theory is a way to answer the above question.  In the
mid-fifties, Pontryagin and Bellman propose generalizations of the calculus of
variations of broad applicability:  the maximum principle and the method of
dynamic programming. Now, these results sustain application in the
biological sciences, and, in particular, to the optimal control
of infectious diseases.

  The approach in this work relies on Pontryagin's Maximum
Principle \cite{Boltyanski1960} and follows the same methodology of
\citet{lenhart2007optimal}. Lenhart's work makes an accessible optimal control
device to describe common epidemic interventions like vaccination, treatment,
quarantine, isolation among others. Our intention in this work is illustrate
the mentioned strategies throughout recent literature and state result for
existence of control solution for a particular family of epidemic models.
Likewise we present  important regarding goals and issues that appear with the
relating theory and numerical approximations. We start by fixing our notation
and enunciate the core of the theory.
