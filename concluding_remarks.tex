\paragraph{\it Uniqueness of optimal policy}
The proof of the uniqueness of the state path $X_u$, given a policy $u$, is
fairly standard (Theorem \ref{ExAdmisPair}). However, the uniqueness of an 
{\it optimal policy} is not trivial and it can be established on some small 
enough interval; see, for instance, \cite{GaffSchaefer09} and the references 
therein.  
\medskip

\paragraph{\it Maximum principle vs. Dynamic programming} 
  The same approach is followed in almost all the related literature on optimal
control of epidemics/diseases. As an alternative, the so-called Dynamic
programming approach can be used to analyze this kind of problems. With the
Maximum principle we need to solve a system of ordinary differential equations
(ODEs) whereas in Dynamic programming a partial differential equation (PDE)
arises. In addition, both approaches involve an optimization problem. The
Maximum principle is mostly used because there are plenty of methods to
numerically solve ODEs.
\medskip
\paragraph{\it Numerical Issues}

\medskip
\paragraph{\it Genetic algorithms}[Frank]