\paragraph{Two-strains Tuberculosis Model.}
	Seeking to reduce the latent and infectious groups with the resistant-strain 
	tuberculosis, in \cite{Lenhart2002} the authors  use controls ro represents 
	two types of treatments in a tuberculosis model which consider the effect of 
	treatment in two kinds of strains. The uncontrolled version is:
	
	\begin{equation}\label{eqn:two_strain_TB}
	  \begin{aligned}
	    \frac{dS}{dt} &=
		    \Lambda - \beta_1 S \frac{I_1}{N} 
		    - \beta^{*} S \frac{I_2}{N}
		    - \mu S
		  \\
		  \frac{L_1}{dt} &=
			  \beta_1 S \frac{I_1}{N}
			  - (\mu + k_1) L_1
			  +  p r_2 I_1
				+ \beta_2 T \frac{I_1}{N}
				- \beta^{*} L_1 \frac{I_2}{N}
			\\
			\frac{I_1}{dt} &= 
				k_1 L_1
				- (\mu + d_1) I_1
				-r_2 I_1
			\\
			\frac{L_2}{dt} &=
				q r_2 I_1
				- (\mu + k_2) L_2
				+ \beta^{*} (S + L_1 + T) \frac{I_2}{N}
			\\
			\frac{I_2}{dt} &=
				k_2 L_2 - (\mu + d_2) I_2
			\\
			\frac{d T}{dt} &=
				r_1 L_1
				+ (1 - (p + q)) r_2 I_1
				- \beta T \frac{I_1}{N}
				- \beta^{*} T \frac{I_2}{N}
				-\mu T ~.
	  \end{aligned}
	\end{equation}

	\citeauthor*{Lenhart2002} consider time dependent 
optimal control strategies associated with case holding and case finding based 
% on the two-strain TB model \eqref{eqn:two_strain_TB}. They incorporates the 
case finding control by adding a term which identifies and cure a fraction o 
latent individuals. This control consequently reduces the rate of disease 
development by latent individuals. The authors includes case holding by adding a 
term which may decrease the treatment failure rate of individuals with sensitive 
TB, so,this control reduce the incidence of drug resistant TB. 

	Using $u_1(t)$ to denote the fraction of typical TB latent individuals that 
is identified and will put under treatment---case finding control--- and
$1 - u_2(t)$ to represent the effort that prevents the failure treatment in 
typical TB infectious individuals, the two-strain-TB model 
\eqref{eqn:two_strain_TB} becomes:
\begin{equation}
	  \begin{aligned}
	    \frac{dS}{dt} &=
		    \Lambda - \beta_1 S \frac{I_1}{N} 
		    - \beta^{*} S \frac{I_2}{N}
		    \mu S
		  \\
		  \frac{L_1}{dt} &=
			  \beta_1 S \frac{I_1}{N}
			  - (\mu + k_1) L_1
			  - u_1 (t) r_1 L_1
			  + (1 - u_2 (t)) p r_2 I_1
				+ \beta_2 T \frac{I_1}{N}
				- \beta^{*} L_1 \frac{I_2}{N}
			\\
			\frac{I_1}{dt} &= 
				k_1 L_1
				- (\mu + d_1) I_1
				-r_2 I_1
			\\
			\frac{L_2}{dt} &=
				(1 - u_2(t)) q r_2 I_1
				- (\mu + k_2) L_2
				+ \beta^{*} (S + L_1 + T) \frac{I_2}{N}
			\\
			\frac{I_2}{dt} &=
				k_2 L_2 - (\mu + d_2) I_2
			\\
			\frac{d T}{dt} &=
				u_1(t) r_1 L_1
				+ (1 - (1 - u_2(t))(p + q)) r_2 I_1
				- \beta T \frac{I_1}{N}
				- \beta^{*} T \frac{I_2}{N}
				-\mu T.
	  \end{aligned}
	\end{equation}

In this context, the controls reduce the latent and infected 
groups with resistant TB. However, case holding and case finding controls 
produces a economic fee. In \cite{Lenhart2002} the authors use
\begin{equation}
	 J(u_1, u_2) =
		 \int_0 ^ {t_f}
			 \left[
				 L_2(t) + I_2(t) 
				 + \frac{B_1}{2} [u_1(t)] ^ 2
				 + \frac{B_2}{2} [u_2(t)] ^ 2
			 \right]dt,
\end{equation}
to describe the regarding cost.
\begin{table}
	\begin{center}
		\begin{tabular}{@{}rll@{}} 
			$S:$
			&
				Susceptible
			\\
			$I:$ 
			&	Infected
			\\
			$R:$ 
			&	Recover
			\\
			\\
			%\toprule
			\multicolumn{1}{c}{Parameter}
			&
			\multicolumn{1}{c}{Meaning}
			& 
			\multicolumn{1}{c}{Value}
			\\
				\midrule
				$\beta$
				& 
					Transmission probability
				&
					---
			\\
				$\gamma$
				&
					Recover rate
			\\
				$\mu$
				&
					Natural death rate
			\\
				$K$
				&
					Carrying capacity
			\\
			\bottomrule
		\end{tabular}
		\caption{Variables and parameters description}
	\end{center}
\end{table}