\subsection*{Two-strains Tuberculosis}
Seeking to reduce the latent and infectious groups with the 
resistant-strain tuberculosis, in \cite{Lenhart2002} the authors  use 
control theory to describe optimal strategies in a tuberculosis model 
which consider {\color{red}(considers?)} the effect of treatment in two kinds of strains. The 
controlled version reads:
	\begin{equation}\label{eqn:MDR-TB_model}
	  \begin{aligned}
	    \frac{dS}{dt} &=
		    \Lambda - \beta_1 S \frac{I_1}{N} 
		    - \beta_3 S \frac{I_2}{N}
		    \mu S,
		  \\
		  \frac{L_1}{dt} &=
			  \beta_1 S \frac{I_1}{N}
			  - (\mu + k_1) L_1
			  - u_1 (t) r_1 L_1
			  + (1 - u_2 (t)) p r_2 I_1
				+ \beta_2 T \frac{I_1}{N}
				- \beta_3 L_1 \frac{I_2}{N},
			\\
			\frac{I_1}{dt} &= 
				k_1 L_1
				- (\mu + d_1) I_1
				-r_2 I_1,
			\\
			\frac{L_2}{dt} &=
				(1 - u_2(t)) q r_2 I_1
				- (\mu + k_2) L_2
				+ \beta_3 (S + L_1 + T) \frac{I_2}{N},
			\\
			\frac{I_2}{dt} &=
				k_2 L_2 - (\mu + d_2) I_2,
			\\
			\frac{d T}{dt} &=
				u_1(t) r_1 L_1
				+ (1 - (1 - u_2(t))(p + q)) r_2 I_1
				- \beta T \frac{I_1}{N}
				- \beta_3 T \frac{I_2}{N}
				-\mu T.
	  \end{aligned}
	\end{equation}

	\citeauthor*{Lenhart2002} consider time dependent 
optimal control strategies associated with \emph{case holding} and 
\emph{case finding}. They incorporates {\color{red} (incorporate?)} the case 
finding control by adding a term which identifies and  cure a fraction o 
{\color{red} (of?)} latent individuals. Case finding consequently reduces the 
rate of disease development by latent individuals. The authors includes 
{\color{red} (include?)} case holding by adding a term which may decrease the 
treatment failure rate of individuals with sensitive  TB, so, this control 
reduce the incidence of drug resistant TB. In model \eqref{eqn:MDR-TB_model}, 
$u_1$ denotes the fraction of typical TB latent individuals that is identified 
and will put under treatment \textemdash case finding control\textemdash and 
$1 - u_2$ represents the effort that prevents the failure treatment in typical 
TB infectious individuals.

	The controls $u_1$ ,$u_2$ reduce the latent and infected 
groups with resistant TB. However, the case holding and the case finding 
strategies produces {\color{red} (produce?)} a economic fee. In \cite{Lenhart2002} the authors use
\begin{equation}\label{eqn:MDR-TB_action}
	 J(u_1, u_2) =
		 \int_0 ^ {t_f}
			 \left[
				 L_2(t) + I_2(t) 
				 + \frac{B_1}{2} [u_1(t)] ^ 2
				 + \frac{B_2}{2} [u_2(t)] ^ 2
			 \right]dt,
\end{equation}
to describe the regarding cost.
%
%
\begin{table}[htb]
  \centering
  \begin{tabular}{rllrl}
    \toprule
       \multicolumn{5}{c}{\textbf{Description}}\\
        \midrule
        $\beta_1$ 
            & Probability that a susceptible 
        &&
          $r_1$ 
            & 
            Treatment recover rate of 
        \\
         & individual becomes infected.
           &&&
            individuals with latent TB.
        \\
        $\beta_2$ 
          & Probability that a recovered 
        && 
          $r_2$ 
            &
            Treatment rate recover of 
          \\
          & individual  become infected
            &&&
              individuals with infectious TB.
          \\
        $\beta_3$ 
          & Probability that a uninfected 
          &&
          $p$, $q$
          & 
            Proportion of infectious individuals 
          \\
          & individual become infected 
            &&&
              that not complete the treatment
            \\
          & by resistant-TB 
            &&&
              for TB or MDR-TB respectively.
      \\
      \\
        $\mu$ 
          & Natural per-capita death rate.
          &&
            $N$ 
            &
              Total population size.
      \\
        $d_1$ 
          & Per-capita death rate by TB.
          &&
            $\Lambda$
            & Recruitment rate.
              
          \\
      $d_2$ 
          & Per-capita  death rate by MDR-TB.
          &&
            $t_f$ 
              & Final time.
          \\
      \\
      $k_1$ 
        & Rate at which an latent TB 
        &&
          $B_1$ 
            &
              Systematic cost of the
        \\
        & individual becomes infectious. 
          &&&
            case finding  control.
      \\
          $k_2$  
          & Rate at which an latent individual
          &&
            $B_2$
            & 
            Cost of the case holding strategy
          \\
          & with MDR-TB becomes infectious.
    \\
    \bottomrule
    \end{tabular}
  \caption{Parameters description and simulation values for the control 
  problem \eqref{eqn:MDR-TB_model}.}
  \label{tbl:parameters_MDR-TB_model_des}
\end{table}

%
