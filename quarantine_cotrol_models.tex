\paragraph{SARS and genetic algorithms. [FRANK]}
Xiefei Yan and Yun Zou report in \cite{Yan2008} a epidemic model for
Severe acute respiratory syndrome (SARS) They use quarantine and isolation as
mitigation controls. The authors propose sub-optimal control policies and
perform numeric simulations with genetic algorithms. Frist, we present the
uncontrolled version used in the mentioned reference and formulated by Gummel 
et. al. in \cite{Gumel2004}:
%
%
\begin{equation}
	\begin{aligned}
		\dfrac{dS}{dt} &=
			\Lambda 
			-\dfrac{
				\beta S
				\left(
					I + \mathcal{E}_E E
					+ \mathcal{Q}_Q Q
					+ \mathcal{J}_J E
				\right)
			}{N}
		\\
		\dfrac{dE}{dt} &=
			p +
			\dfrac{
				\beta S
				\left(
					I + \mathcal{E}_E E
						+ \mathcal{Q}_Q Q
						+ \mathcal{J}_J E
				\right)
			}{N}
			-(\gamma + \kappa_1 + \mu) E
		\\
		\dfrac{dQ}{dt} &=
			\gamma_1 E 
			- (\kappa_2 + \mu) Q
		\\
		\dfrac{dI}{dt} &=
			\kappa_1 E 
			-(\gamma_2 + d_1  + \sigma_1 + \mu I)
		\\
		\dfrac{dJ}{dt} &=
			\gamma_2 I 
			+ \kappa_2 Q
			- (\sigma_2 + d_2 + \mu) J
		\\
		\dfrac{dR}{dt} &=
			\sigma_1 I
			+\sigma_2 J
			- \mu R.
	\end{aligned}
\end{equation}
