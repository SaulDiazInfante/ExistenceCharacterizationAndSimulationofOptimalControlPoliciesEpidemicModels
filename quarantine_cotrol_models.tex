
\subsection*{SARS}
\citeauthor{Yan2008} report in \cite{Yan2008} a epidemic model for
Severe acute respiratory syndrome (SARS). They use quarantine and 
isolation as mitigation controls. The authors propose sub-optimal 
control policies and perform numeric simulations with genetic 
algorithms. The controlled version used in the mentioned 
reference reads:
%
%
\begin{equation}\label{eqn:sars_model}
	\begin{aligned}
		\dfrac{dS}{dt} &=
			\Lambda 
			-\dfrac{
				S
				\left(
					\beta I 
					+ \mathcal{E}_E  \beta E
					+ \mathcal{E}_Q  \beta Q
					+ \mathcal{E}_J  \beta J
				\right)
			}{N}
			- \mu S,
		\\
		\dfrac{dE}{dt} &=
			p +
			\dfrac{
				\beta S
				\left(
					\beta I 
						+ \mathcal{E}_E \beta E
						+ \mathcal{E}_Q \beta Q
						+ \mathcal{E}_J \beta J
				\right)
			}{N}
			-(
				u_1(t) + k_1 + \mu
			)E,
		\\
		\dfrac{dQ}{dt} &=
			u_1(t) E 
			- (k_2 + \mu) Q
		\\
		\dfrac{dI}{dt} &=
			k_1 E 
			-(u_2(t) + d_1  + \sigma_1 + \mu) I,
		\\
		\dfrac{dJ}{dt} &=
			u_2(t) I 
			+ k_2 Q
			- (d_2 + \sigma_2 + \mu) J,
		\\
		\dfrac{dR}{dt} &=
			\sigma_1 I
			+\sigma_2 J
			- \mu R.
	\end{aligned}
\end{equation}
The control variable $u_1$ denotes the proportion of quarantining people 
who had contact with an infected person inside of a quarantining program or
educational campaigns. Control $u_2$ models the proportion of symptomatic 
population which are in a isolation program. The authors consider the 
following epidemiological classes.
\begin{table}[h!]
	\begin{center}
		\begin{tabular}{@{}rll@{}} 
			$S$: & Susceptible individuals 
			\\
			$E$: & Asymptomatic individuals who have been 
			\\
			   & exposed to the virus but have not yet developed 
			\\
			   & clinical symptoms of SARS 
			\\
			$Q$: & Quarantine individuals
			\\
			$I$: & Symptomatic 
			\\
			$J$: & Isolated
			\\
			$R$: & Recovered
			\\
				& $N = S + E + Q + I + J + R$.
		\end{tabular}
	\end{center}
\end{table}
Description of parameters and its values are enclosed 
in \Cref{tbl:sars_table}.
So, giving the disease dynamics in \eqref{eqn:sars_model}, the problem is to 
minimize the functional cost
\begin{equation}
  J(u_1,u_2)
    = \int_{0}^{t_f}
      \left[
        B_1 E(t)
        + B_2 Q(t)
        + B_3 I(t)
        + B_4 J(t)
        + \frac{C_1}{2} u_1^2 (t)
        + \frac{C_2}{2} u_2^2 (t)
      \right]
      dt.
\end{equation}
%
\begin{table}[htb]
    \begin{center}
      \begin{tabular}{@{}rl@{}}
        \toprule
        & \multicolumn{1}{l}{\bf{Description}}
        \\
        \midrule
        $\beta$
          & Transmission coefficient
        \\
        $\varepsilon_E$, 
        $\varepsilon_Q$,
        $\varepsilon_J$
          & Modification parameter for 
          \\
          &  exposed, quarantine and isolation classes 
          \\
        $\mu$
          & Natural death rate.
        \\
        $\Lambda$
          & Constant recruitment rate
        \\
        $p$
          & Net inflow of asymptomatic individuals
        \\
        $k_1$ 
          & Transfer rate from class 
          \\
          & of asymptomatic to symptomatic
          \\
        $k_2$
          & Transfer rate from the quarantine 
          \\ 
          & class to isolation
        \\
        \\
        $d_1$, $d_2$
          & Per-capita disease induced death rates 
          \\
          & for the symptomatic individuals and 
          \\
          & isolated individuals.
        \\
        $\sigma_1$, $\sigma_2$
          & Per-capita recovery rates for the 
          \\
          & symptomatic individuals and 
          \\
          &  isolated individuals
        \\
        \\
        $t_f$
          & Final time 
        \\
        $B_1$, $B_2$, $B_3$, $B_4$
        & Respectively cost for 
        \\
        &
          $E$,$Q$,$I$,$J$ classes
        \\
        $C_1$, $C_2$
        & Costs for Isolation and Quarantine 
        \\
          & policies.
        \\
        \bottomrule
      \end{tabular}
     \caption{Parameter description for the SARS model
     \eqref{eqn:sars_model}.}
     \label{tbl:sars_table}
     \end{center}
\end{table}