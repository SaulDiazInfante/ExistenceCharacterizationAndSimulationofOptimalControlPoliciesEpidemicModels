%The models described in the above sections can be 
\rm 




In this section we consider a family of control problems that include the epidemic models described above.  Existence of a unique solution, given any control policy; existence of optimal policies; and characterization of such optimal policies via the Maximum principle.

Let $\mathbf{X}\s\R^n$ and $\mathbf{U}\s \R^m$ be nonempty sets. The sets $\mathbf{X}$ and $\mathbf{U}$ are 
respectively called the {\it state space} and the {\it control space}. Consider 
the following {\it control system}
\begin{equation}\label{CoDiffEq}
\dot{X}(t)=f(t,X(t),u(t)), \quad X(0)=x_0,
\end{equation}
where $f:[0,T]\times \mathbf{X}\times \mathbf{U}\to \R^n$ and $u:[0,T]\to U$. 



\subsection{The state path given any control policy}

%In order to guarantee the existence of a solution $x$ to \eqref{CoDiffEq}, 
%we need the following.


%A proof of the following result can be found, for instance, in Yong \cite[Sect. 2.1]{Yong2015}. 

\begin{theorem}\label{ExAdmisPair} Assume the function $f:[0,T]\times \mathbf{X}\times \mathbf{U}\to \R^n$ in the system \eqref{CoDiffEq} is measurable and there exists a constant $L>0$ such that
\begin{eqnarray}
  \|f(t,x,u)-f(t,x_1,u)\| & \leq & L\|x-x_1\|\label{LipfInx}\\
  \|f(t,0,u)\| & \leq & L\label{fBound}
\end{eqnarray}
for every $x,x_1\in \mathbf{X}$, $t\in[0,T]$, and $u\in \mathbf{U}$.

	Under Assumption \ref{Assum1}, for each measurable function 
	$u:[0,T]\to \mathbf{U}$, there exists a unique absolutely continuous function $X_u$ that satisfies the the system 
	\eqref{CoDiffEq} almost everywhere. %The solution $x_u$ satisfies	\begin{equation}\label{ineqSol} 
	%	|x_u(t)|\leq e^{Lt}(1+|x_0|) -1,\qquad t\in [0,T].	\end{equation}
\end{theorem}
\begin{proof} Let $u:[0,T]\to U$ be a measurable function. Consider the linear space 
    \[\mathbb{X}=\{X:[0,T]\to \R^n\mid X \mbox{ is continuous}\}\] 
with the norm
    \[ \|X\|_w:=\sup_{t\in[0,T]} \frac{\|X(t)\|}{w(t)}, \]
where $w(t):=e^{Lt}$ for each $t\in [0,T]$. It can be shown, with slight modifications in \cite[Section 2.1]{Teschl}, that the pair $(\mathbb{X},\|\cdot\|_w)$ is a Banach space. Define the operator $K:\mathbb{X}\to \mathbb{X}$ by 
    \[ K[X](t):=x_0 + \int_0^t f(s,X(s),u(s))ds.\]
By \eqref{LipfInx} and \eqref{fBound}, any $(t,x,u)$ satisfies 
  \begin{equation}
      \|f(t,x,u)\| \leq  L(1+\|x\|),
  \end{equation}
thus $f(\cdot,X(\cdot),u(\cdot))$ is Lebesgue integrable and $K[X]$ is absolutely continuous. We claim that $K$ is a contraction with contraction constant $1-e^{-LT}$. Indeed,
    \begin{eqnarray*}
    \| K[X] - K[Y] \|_w & = & \sup_{t\in[0,T]} \frac{|\int_0^t [f(s,X(s),u(s)) -f(s,Y(s),u(s))]ds|}{w(t)}\\
        & \leq &   \sup_{t\in[0,T]} \frac{L\int_0^t w(s)[w(s)]^{-1}|X(s) -Y(s)|ds}{w(t)}\\
        &\leq &  L\|X-Y\|_w \sup_{t\in[0,T]} \frac{\int_0^t w(s)ds}{w(t)}\\
        & = &  L\|X-Y\|_w \sup_{t\in[0,T]}\frac{[e^{Lt}-1]/L}{e^{Lt}}\\
        & = &  (1-e^{-LT})\|X-Y\|_w. 
    \end{eqnarray*}
Then by Banach's fixed point theorem \cite[Theorem 2.1]{Teschl}, there exists a unique $X\in \mathbb{X}$ satisfying 
    \[ X(t)=x_0 + \int_0^t f(s,X(s),u(s))ds.\]
Therefore \eqref{CoDiffEq} holds almost everywhere \cite[Corollary 5.4.1]{Loeb2016}.
\end{proof}



\subsection{Existence of optimal control policies}



\subsection{Characterization of optimal policies}
