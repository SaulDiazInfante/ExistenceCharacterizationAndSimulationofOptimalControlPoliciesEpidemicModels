\paragraph{Standard SIR model with logistic grow.}
We use the standard population model with logistic grow reported
\cite{Schaefer2009}, see table for parameter description. To introduce the 
vaccination and treatment as mitigation control policies, we firs deal with the
uncontrolled dynamics described by:
\begin{equation}\label{eqn:SIR}
	\begin{aligned}
		\frac{dS}{dt} &=
			\mu N  
			- \beta \frac{S I}{N} 
			- \mu \frac{N}{K} S 
		\\
		\frac{dI}{dt} &=
			\beta \frac{S I}{N}
			- \gamma I
			- \mu \frac{N}{K} I
		\\
		\frac{dR}{dt} &= 
			\gamma I 
			- \mu \frac{N}{K} R 
		\\
		S(0) &= S_0, \quad
		I(0) = I_0, \quad
		R(0) = R_0
		\\
		N &= S + I +R
		\\
		\frac{dN}{dt} &=
			\mu N 
			\left(
				1 - \frac{N}{K}
			\right).
	\end{aligned}
\end{equation}

Given initial population sizes $S_0, I_0, R_0$, the goal is to determine the best policy to
mitigate the outbreak described by the SIR model \eqref{eqn:SIR} and optimize a regarding  cost. In this context, the policies are Lebesgue measurable bounded functions
which optimize a given convex functional cost. For example, \citeauthor{Schaefer2009} pursue to minimize the infected population described by \eqref{eqn:SIR} while minimizing the cost of vaccination $(u_1(t))$ and treatment ($u_2(t))$. In symbols, the authors seek to minimize the objective functional
\begin{equation}
	\int_{0}^T	
		\left[
			B_1 I(t) 
			+ B_2 \left[\frac{R(t)}{K}\right]^m u_1(t)^2 + B_3 u_2(t)
		\right],
		\qquad  m\geq 1,
\end{equation}
subject to
\begin{equation}
	\begin{aligned}
		\frac{dS}{dt} &=
			\mu N  
			- \beta \frac{S I}{N} 
			- \mu \frac{N}{K} S - u_1(t) S
		\\
		\frac{dI}{dt} &=
			\beta \frac{S I}{N}
			- (\gamma  + \mu) I 
			- \mu \frac{N}{K} I
			- u_2(t) I
		\\
		\frac{dR}{dt} &= 
			\gamma I 
			- \mu \frac{N}{K} R 
			+ u_1(t) S 
			+ u_2(t) I
		\\
		S(0) &= S_0, \quad
		I(0) = I_0, \quad
		R(0) = R_0. \quad
	\end{aligned}
\end{equation}