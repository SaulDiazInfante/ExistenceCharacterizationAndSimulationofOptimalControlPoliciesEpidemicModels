\subsection{Deterministic OCPs in continuous time}

Let $\mathbf{X}\s\R^n$ and $\mathbf{U}\s \R^m$ be nonempty sets. The sets $\mathbf{X}$ and $\mathbf{U}$ are 
respectively called the {\it state space} and the {\it control space}. Consider 
the following {\it control system}
\begin{equation}\label{CoDiffEq}
\dot{X}(t)=f(t,X(t),u(t)),\qquad t\in[0,T], \quad X(0)=x_0.
\end{equation}
where $f:[0,T]\times \mathbf{X}\times \mathbf{U}\to \R^n$ and $u:[0,T]\to U$. 
%In order to guarantee the existence of a solution $x$ to \eqref{CoDiffEq}, 
%we need the following.

\begin{assumption}\label{Assum1}  The function $f:[0,T]\times \mathbf{X}\times \mathbf{U}\to \R^n$
is measurable and there exists a constant $L>0$ such that
\begin{eqnarray}
  \|f(t,x,u)-f(t,x_1,u)\| & \leq & L\|x-x_1\|\label{LipfInx}\\
  \|f(t,0,u)\| & \leq & L\label{fBound}
\end{eqnarray}
for every $x,x_1\in \mathbf{X}$, $t\in[0,T]$, and $u\in \mathbf{U}$.
\end{assumption}

%A proof of the following result can be found, for instance, in Yong \cite[Sect. 2.1]{Yong2015}. 

\begin{theorem}\label{ExAdmisPair} 
	Under Assumption \ref{Assum1}, for each measurable function 
	$u:[0,T]\to \mathbf{U}$, there exists a unique absolutely continuous function $X_u$ that satisfies the the system 
	\eqref{CoDiffEq} almost everywhere. %The solution $x_u$ satisfies	\begin{equation}\label{ineqSol} 
	%	|x_u(t)|\leq e^{Lt}(1+|x_0|) -1,\qquad t\in [0,T].	\end{equation}
\end{theorem}
\begin{proof} Let $u:[0,T]\to U$ be a measurable function. Consider the linear space 
    \[\mathbb{X}=\{X:[0,T]\to \R^n\mid X \mbox{ is continuous}\}\] 
with the norm
    \[ \|X\|_w:=\sup_{t\in[0,T]} \frac{|X(t)|}{w(t)}, \]
where $w(t):=e^{Lt}$ for each $t\in [0,T]$. It can be shown, with slight modifications in \cite[Section 2.1]{Teschl}, that the pair $(\mathbb{X},\|\cdot\|_w)$ is a Banach space. Define the operator $K:\mathbb{X}\to \mathbb{X}$ by 
    \[ K[X](t):=x_0 + \int_0^t f(s,X(s),u(s))ds.\]
By \eqref{LipfInx} and \eqref{fBound}, any $(t,x,u)$ satisfies 
  \begin{equation}
      \|f(t,x,u)\| \leq  L(1+\|x\|),
  \end{equation}
thus $f(\cdot,X(\cdot),u(\cdot))$ is Lebesgue integrable and $K[X]$ is absolutely continuous. We claim that $K$ is a contraction with contraction constant $1-e^{-LT}$. Indeed,
    \begin{eqnarray*}
    \| K[X] - K[Y] \|_w & = & \sup_{t\in[0,T]} \frac{|\int_0^t [f(s,X(s),u(s)) -f(s,Y(s),u(s))]ds|}{w(t)}\\
        & \leq &   \sup_{t\in[0,T]} \frac{L\int_0^t w(s)[w(s)]^{-1}|X(s) -Y(s)|ds}{w(t)}\\
        &\leq &  L\|X-Y\|_w \sup_{t\in[0,T]} \frac{\int_0^t w(s)ds}{w(t)}\\
        & = &  L\|X-Y\|_w \sup_{t\in[0,T]}\frac{[e^{Lt}-1]/L}{e^{Lt}}\\
        & = &  (1-e^{-LT})\|X-Y\|_w. 
    \end{eqnarray*}
Then by Banach's fixed point theorem \cite[Theorem 2.1]{Teschl}, there exists a unique $X\in \mathbb{X}$ satisfying 
    \[ X(t)=x_0 + \int_0^t f(s,X(s),u(s))ds.\]
Therefore \eqref{CoDiffEq} holds almost everywhere \cite[Corollary 5.4.1]{Loeb2016}.
\end{proof}


	Given a measurable function $u$ and the solution $X_u(\cdot)$ to 
\eqref{CoDiffEq}, the terminal state $X_u(T)$ can be constrained to belong to a set $M\s X$, in such a case we assume the following. 

\begin{assumption} Let $M$ be a nonempty subset of $\mathbf{X}$. 
	The set of {\it admissible controls} 
	\begin{equation}\label{FeasCont}  
		 \mathbb{U}_M:=\{u:[0,T]\to \mathbf{U}\mid u\  
		 \mbox{\rm is measurable and } X_u(T)\in M\} 
	\end{equation}
	is nonempty. A pair $(u,X_u)$, where $u\in \mathbb{U}_M$, is called an 
	{\it admissible pair}. To ease notation, we simply write $(u,X)$.
\end{assumption}


Consider the {\it performance index} or {\it cost functional} of an admissible control $u$, given the initial state $x_0$, 
        \begin{equation}\label{PiBolza} V(u,x_0) := \int_0^Tg(t,X(t),u(t))\,dt + h(X(T)),\end{equation}
where $g:[0,T]\times \mathbf{X}\times \mathbf{U}\to \R$ and $h:\mathbf{X}\to\R$ are measurable.

The {\it optimal control problem} (OCP) consists of finding an admissible control $u^\ast$ such that
\[ V(u^\ast,x_0)=\sup\{ V(u,x_0)\mid u\in \mathbb{U}_M \}.\]
If there exists such a control $u^\ast$, then it is called an {\it optimal policy} or {\it optimal control}. The pair $(X^\ast,u^\ast)$, where $X^\ast$ is given by Theorem \ref{ExAdmisPair}, is called an {\it optimal pair}.



\begin{remark}\rm
The performance index \eqref{PiBolza} is said to be in the {\it Bolza form}. When $g= 0$ and $h\neq 0$, it is said to be in the {\it Mayer form}. Another form occurs when $h= 0$ and $g\neq 0$; in such a case, \eqref{PiBolza} is said to be in the {\it Lagrange form}. These three forms are equivalent; see, for instance, Cesari \cite[Sect. 1.9]{Cesari83}. 
\end{remark} 

%%------------------------------------------------------------------------------

\subsection{Existence of optimal policies}




\begin{assumption}
The functions $g$ and $h$ in the performance index \eqref{PiBolza}
 
\end{assumption}










%%------------------------------------------------------------------------------
\subsection{The maximum principle}


The following lemma is a classical result in Calculus of Variations. We give a proof for completness.

\begin{lemma}\label{VariatLemma} Let  

\end{lemma}




%%------------------------------------------------------------------------------
\subsection{Sufficient conditions}






