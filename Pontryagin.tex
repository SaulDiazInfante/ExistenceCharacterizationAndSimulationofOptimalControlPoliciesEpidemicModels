%We follow Cesari \cite{Cesari83} and Yong \cite{Yong2015} for OCPs. We assume some knowledge on ODEs \cite{Teschl} and real analysis \cite{Loeb2016}.

\subsection{Deterministic OCPs in continuous time}

Let $\mathbf{X}\s\R^n$ and $\mathbf{U}\s \R^m$ be nonempty sets. The sets $\mathbf{X}$ and $\mathbf{U}$ are 
respectively called the {\it state space} and the {\it control space}. Consider 
the following {\it control system}
\begin{equation}\label{CoDiffEq}
\dot{X}(t)=f(t,X(t),u(t)), \quad X(0)=x_0,
\end{equation}
where $f:[0,T]\times \mathbf{X}\times \mathbf{U}\to \R^n$ and $u:[0,T]\to U$. 
%In order to guarantee the existence of a solution $x$ to \eqref{CoDiffEq}, 
%we need the following.

\begin{assumption}\label{Assum1}\rm  The function $f:[0,T]\times \mathbf{X}\times \mathbf{U}\to \R^n$
is measurable and there exists a constant $L>0$ such that
\begin{eqnarray}
  \|f(t,x,u)-f(t,x_1,u)\| & \leq & L\|x-x_1\|\label{LipfInx}\\
  \|f(t,0,u)\| & \leq & L\label{fBound}
\end{eqnarray}
for every $x,x_1\in \mathbf{X}$, $t\in[0,T]$, and $u\in \mathbf{U}$.
\end{assumption}

%A proof of the following result can be found, for instance, in Yong \cite[Sect. 2.1]{Yong2015}. 

\begin{theorem}\label{ExAdmisPair} 
	Under Assumption \ref{Assum1}, for each measurable function 
	$u:[0,T]\to \mathbf{U}$, there exists a unique absolutely continuous function $X_u$ that satisfies the the system 
	\eqref{CoDiffEq} almost everywhere. %The solution $x_u$ satisfies	\begin{equation}\label{ineqSol} 
	%	|x_u(t)|\leq e^{Lt}(1+|x_0|) -1,\qquad t\in [0,T].	\end{equation}
\end{theorem}
\begin{proof} Let $u:[0,T]\to U$ be a measurable function. Consider the linear space 
    \[\mathbb{X}=\{X:[0,T]\to \R^n\mid X \mbox{ is continuous}\}\] 
with the norm
    \[ \|X\|_w:=\sup_{t\in[0,T]} \frac{\|X(t)\|}{w(t)}, \]
where $w(t):=e^{Lt}$ for each $t\in [0,T]$. It can be shown, with slight modifications in \cite[Section 2.1]{Teschl}, that the pair $(\mathbb{X},\|\cdot\|_w)$ is a Banach space. Define the operator $K:\mathbb{X}\to \mathbb{X}$ by 
    \[ K[X](t):=x_0 + \int_0^t f(s,X(s),u(s))ds.\]
By \eqref{LipfInx} and \eqref{fBound}, any $(t,x,u)$ satisfies 
  \begin{equation}
      \|f(t,x,u)\| \leq  L(1+\|x\|),
  \end{equation}
thus $f(\cdot,X(\cdot),u(\cdot))$ is Lebesgue integrable and $K[X]$ is absolutely continuous. We claim that $K$ is a contraction with contraction constant $1-e^{-LT}$. Indeed,
    \begin{eqnarray*}
    \| K[X] - K[Y] \|_w & = & \sup_{t\in[0,T]} \frac{|\int_0^t [f(s,X(s),u(s)) -f(s,Y(s),u(s))]ds|}{w(t)}\\
        & \leq &   \sup_{t\in[0,T]} \frac{L\int_0^t w(s)[w(s)]^{-1}|X(s) -Y(s)|ds}{w(t)}\\
        &\leq &  L\|X-Y\|_w \sup_{t\in[0,T]} \frac{\int_0^t w(s)ds}{w(t)}\\
        & = &  L\|X-Y\|_w \sup_{t\in[0,T]}\frac{[e^{Lt}-1]/L}{e^{Lt}}\\
        & = &  (1-e^{-LT})\|X-Y\|_w. 
    \end{eqnarray*}
Then by Banach's fixed point theorem \cite[Theorem 2.1]{Teschl}, there exists a unique $X\in \mathbb{X}$ satisfying 
    \[ X(t)=x_0 + \int_0^t f(s,X(s),u(s))ds.\]
Therefore \eqref{CoDiffEq} holds almost everywhere \cite[Corollary 5.4.1]{Loeb2016}.
\end{proof}


	Given a measurable function $u$ and the solution $X_u(\cdot)$ to 
\eqref{CoDiffEq}, the terminal state $X_u(T)$ can be constrained to belong to a set $M\s X$, in such a case we assume the following. 

\begin{assumption}\rm Let $\mathbf{B}$ be a nonempty subset of $\mathbf{X}$. 
	The set of {\it admissible controls} 
	\begin{equation}\label{FeasCont}  
		 \mathbb{U}_\mathbf{B}:=\{u:[0,T]\to \mathbf{U}\mid u\  
		 \mbox{\rm is measurable and } X_u(T)\in \mathbf{B}\} 
	\end{equation}
	is nonempty. A pair $(u,X_u)$, where $u\in \mathbb{U}_\mathbf{B}$, is called an 
	{\it admissible pair}. To ease notation, we simply write $(u,X)$.
\end{assumption}


Consider the {\it performance index} (or {\it cost functional} if we want to minimize) of an admissible control $u$, given the initial state $x_0$, 
        \begin{equation}\label{PiBolza} V(u,x_0) := \int_0^Tg(t,X(t),u(t))\,dt + h(X(T)),\end{equation}
where $g:[0,T]\times \mathbf{X}\times \mathbf{U}\to \R$ and $h:\mathbf{X}\to\R$ are measurable.

The {\it optimal control problem} (OCP) consists of finding an admissible control $u^\ast$ such that
\[ V(u^\ast,x_0)=\sup\{ V(u,x_0)\mid u\in \mathbb{U}_B \}.\]
If there exists such a control $u^\ast$, then it is called an {\it optimal policy} or {\it optimal control}. The pair $(u^\ast,X^\ast)$, where $X^\ast$ is given by Theorem \ref{ExAdmisPair}, is called an {\it optimal pair}.

The elements of the OCP are given in the following seven-tuple 
   \begin{equation}\label{OCP}
       (\mathbf{X},\mathbf{U},\mathbf{B},f,g,h,T).
   \end{equation}



\begin{remark}\rm
The performance index \eqref{PiBolza} is said to be in the {\it Bolza form}. When $g= 0$ and $h\neq 0$, it is said to be in the {\it Mayer form}. Another form occurs when $h= 0$ and $g\neq 0$; in such a case, \eqref{PiBolza} is said to be in the {\it Lagrange form}. These three forms are equivalent; see, for instance, Cesari \cite[Sect. 1.9]{Cesari83}. 
\end{remark} 

%%------------------------------------------------------------------------------

\subsection{Existence of optimal policies}

\begin{assumption}\rm The sets $\mathbf{X}$,  $\mathbf{U}$, and $\mathbf{B}$ are compact. The functions $f$, $g$, and $h$ are continuous.
\end{assumption}
A proof of the following theorem can be found, for instance, in Cesari \cite[Sect. 9.3.]{Cesari83} or Yong \cite[Theorem 2.2.1]{Yong2015}. 
\begin{theorem}[Filippov] Suppose that Assumptions 1, 2, and 3 hold. If, for almost every $t$ in $[0,T]$, each set
        \begin{equation}\label{convexF(t,x)}
F(t,x):= \{ (\alpha, f(t,x,u))\in \mathbb{R}\times \mathbb{R}^n\mid  
    \alpha \leq g(t,x,u), \  u\in \mathbf{U}\},\qquad x\in X,
        \end{equation}
is convex, then there exists an optimal pair $(u^\ast,X^\ast)$.
\end{theorem} 
Next we give sufficient conditions for the set $F(t,x)$ to be convex. The epidemic models considered in this paper satisfy such conditions.

\begin{remark}\rm Suppose the set $\mathbf{U}$ is convex and, for each $(t,x)$, 
\begin{enumerate}[\rm (a)]
    \item the function $g(t,x,\cdot)$ is concave, i.e.,
        \[ g(t,x,\alpha u+(1-\alpha)u') \geq \alpha g(t,x,u) +(1-\alpha) g(t,x,u') \quad \forall u,u'\in\mathbf{U},\  \alpha\in (0,1), \]
       
    \item the function $f(t,x,\cdot)$ is affine, i.e., 
      \[ f(t,x,\alpha u+(1-\alpha)u') = \alpha f(t,x,u) +(1-\alpha) f(t,x,u') \quad \forall  u,u'\in\mathbf{U},\   \alpha\in (0,1). \]
\end{enumerate} 
Then the set $F(t,x)$ in \eqref{convexF(t,x)} is convex.
\end{remark}








%%------------------------------------------------------------------------------
\subsection{The maximum principle}


Define the {\it Hamiltonian} for $(t,x,u,\lambda_0,\lambda)$ in $[0,T]\times \mathbf{X}\times \mathbf{U}\times\R\times\R^n$
    \[H(t,x,u,\lambda_0,\lambda):= \lambda_0g(t,x,u) + \lambda^\top f(t,x,u).  \]  
    

\begin{assumption}\rm  %In addition to Assumptions 1, 2, and 3, the Hamiltonian satisfies the following.
\begin{enumerate}[(a)]
    \item The function $h$ is of class $\mathcal{C}^1$.
%    \item For every $(u,x,\lambda_0,\lambda)$, the function $H(\cdot,x,u,\lambda_0,\lambda)$ is measurable.
    \item For every $(t,u,\lambda_0,\lambda)$, the function $H(t,\cdot,u,\lambda_0,\lambda)$ is of class $\mathcal{C}^1$.
    \item For every $(t,x,\lambda_0,\lambda)$, the functions
     \[ H(t,x,\cdot,\lambda_0,\lambda) \mbox{ and }   H_x(t,x,\cdot,\lambda_0,\lambda)  \]
are continuous.
\end{enumerate}
\end{assumption}


The following theorem is proved in Yong \cite[Theorem 2.3.1]{Yong2015}.

\begin{theorem}[Maximum Principle] Suppose Assumptions 3 and 4 hold and the set $\mathbf{B}$ is convex. Let $(u^\ast,X^\ast)$ be an optimal pair. Then there exists a constant $\lambda_0\geq 0$ and an absolutely continuous function $\lambda:[0,T]\to\R^n$, with
    \begin{equation}\label{RegConL}
        (\lambda_0)^2 + \|\lambda(T)-\lambda_0h_x(X^\ast(T))^\top\|^2 = 1,
    \end{equation}
that satisfy
\begin{enumerate}[\rm (a)]
    \item the maximum condition, for almost every $t\in[0,T]$, 
        \begin{equation}\label{MaxCond}
             H(t,X^\ast(t),u^\ast(t),\lambda_0,\lambda(t)) \geq  H(t,X^\ast(t),u,\lambda_0,\lambda(t)) \quad \forall u\in \mathbf{U},
        \end{equation}
\item the adjoint equation, for almost every $t\in[0,T]$, 
      \begin{equation}\label{AdjEq}
          \dot{\lambda}(t) = -H_x(t,X^\ast(t),u^\ast(t),\lambda_0,\lambda(t))^\top,  
      \end{equation}
\item and the transversality condition
   \begin{equation}\label{TransCond}
    [\lambda(T)^\top-\lambda_0h_x(X^\ast(T))] [y-X^\ast(T)]\geq 0 \quad \forall y\in \mathbf{B}.
   \end{equation}
\end{enumerate}
\end{theorem}


\begin{remark}\label{RemNoCons}\rm
As pointed out by Yong \cite[p. 43]{Yong2015}, if $\mathbf{B}=\R^n$, then \eqref{TransCond} implies 
  \[\lambda(T)-\lambda_0h_x(X^\ast(T))^\top=0\]
  and so $\lambda_0=1$ by \eqref{RegConL}. In such a case, the Hamiltonian takes the form  
      \[\mathcal{H}(t,x,u,\lambda):= g(t,x,u) + \lambda^\top f(t,x,u) = H(t,x,u,1,\lambda)  \]  
    and the adjoint equation \eqref{AdjEq} and the transversality condition \eqref{TransCond} become 
  %\begin{equation}
    \[    \dot{\lambda}(t) = -g_x(t,X^\ast(t),u^\ast(t))^\top -  [f_x(t,X^\ast(t),u^\ast(t))]^\top \lambda(t), \quad \lambda(T)=h_x(X^\ast(T))^\top.
  \]
  % \end{equation} 
  \end{remark}
 


%%------------------------------------------------------------------------------
\subsection{Sufficient conditions}


Consider the OCP \eqref{OCP} with $\mathbf{B}=\R^n$ and $h\equiv 0$. Define
\begin{eqnarray*}
\mathcal{H}^\ast(t,x,\lambda) & := & \sup_{u\in\mathbf{U}}\mathcal{H}(t,x,u,\lambda)\\
     & = & \sup_{u\in\mathbf{U}}\{g(t,x,u) + \lambda^\top f(t,x,u)\}.    
\end{eqnarray*}

\begin{assumption}\label{ConcavityH}\rm The functions $\mathcal{H}$ and $\mathcal{H}_x$ are continuous. 
\end{assumption}

\begin{assumption}\label{piecewise}\rm The functions $u^\ast:[0,T]\to\mathbf{U}$ and $X^\ast:[0,T]\to\mathbf{X}$ satisfy the following
\begin{enumerate}[(a)]
    \item $u^\ast$ is continuous on $[0,T]$ except at a finite number of points 
    \item if $u^\ast$ is discontinuous at $t$, then  
        \[\lim_{s\to t^-}u^\ast(s) \mbox{ and } \lim_{s\to t^+}u^\ast(s)\]
        are finite,
    \item $X^\ast$ is continuous on $[0,T]$, 
    \item $\dot{X}^\ast$ exists and it is continuous on $[0,T]$ except at a finite number of points,
    \item  if $\dot{X}^\ast$ is discontinuous at $t$, then  
        \[\lim_{s\to t^-}\dot{X}(s) \mbox{ and } \lim_{s\to t^+}\dot{X}^\ast(s)\]
        are finite. 
\end{enumerate}
\end{assumption}


The following theorem is proved in Seierstad and Syds\ae ter \cite[Theorem 3]{SeiSyd77}.  

\begin{theorem}\label{SufficientCond} Suppose that Assumption \ref{ConcavityH} holds. Let $(u^\ast,X^\ast)$ be an admissible pair that satisfies Assumption \ref{piecewise}. Suppose that there exists a continuous function $\lambda:[0,T]\to \R^n$ such that 
   \begin{equation}
       \mathcal{H}(t,X^\ast(t),u^\ast(t),\lambda(t)) \geq
       \mathcal{H}(t,X^\ast(t),u,\lambda(t)) \quad \forall u\in\mathbf{U},
   \end{equation}
and, except at the points of discontinuity of $u^\ast$,
    \begin{equation}
         \dot{\lambda}(t) = -\mathcal{H}_x(t,X^\ast(t),u^\ast(t),\lambda_0,\lambda(t))^\top, \quad \lambda(T)=0.
    \end{equation}
If the set $\mathbf{X}$ is convex and, for each $(t,\alpha)$, the function $\mathcal{H}^\ast(t,\cdot,\alpha)$ is concave on $\mathbf{X}$, then $(u^\ast,X^\ast)$ is an optimal pair.
\end{theorem}

%\begin{assumption} The sets $\mathbf{X}$ and $\mathbf{U}$ are convex. The function $\mathcal{H}^\ast(t,\cdot,\lambda)$ is concave\end{assumption}





