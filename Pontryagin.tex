  Consider the following {\it control system}
\begin{equation}\label{GenContSyst}
\dot{X}(t)=f(t,X(t),u(t)), \quad X(0)=x_0,\quad 0\leq t\leq T,
\end{equation}
where $f:[0,T]\times \mathbf{X}\times \mathbf{U}\to \R^n$ and $u:[0,T]\to U$. For each $u$, it is assumed that there exists a unique solution $X_u$ to \eqref{GenContSyst} (ensured, for instance, by Theorem \ref{ExAdmisPair}). In some applications the terminal state $X_u(T)$ is constrained to belong to a given set $\mathbf{B}$, then the set of {\it admissible controls} is defined
	\begin{equation}\label{FeasCont}  
		 \mathbb{U}_\mathbf{B}:=\{u:[0,T]\to \mathbf{U}\mid u\  
		 \mbox{\rm is measurable and } X_u(T)\in \mathbf{B}\} 
	\end{equation}
and $\mathbb{U}_\mathbf{B}$	is also assumed nonempty. A pair $(u,X_u)$, where $u\in \mathbb{U}_\mathbf{B}$, is called an 
	{\it admissible pair}. To ease notation, we simply write $(u,X)$.



The following performance index  %\eqref{PiBolza}
is said to be in the {\it Bolza form}
\begin{equation}\label{PiBolza}
    V(u,x_0):=\int_0^Tg(t,X(t),u(t))dt + h(X(T)),
\end{equation}
where $g:[0,T]\times \mathbf{X}\times \mathbf{U}\to R$ and $h:\mathbf{X}\to \R$ are measurable. When $g= 0$ and $h\neq 0$, it is said to be in the {\it Mayer form}. Another form occurs when $h= 0$ and $g\neq 0$; in such a case, \eqref{PiBolza} is said to be in the {\it Lagrange form}. These three forms are equivalent; see, for instance, Cesari \cite[Sect. 1.9]{Cesari83}. 

In Section X.X we considered minimization problems, in contrast, in this appendix we consider maximization problems. The reason  is due to the name {\it Maximum principle} which appears in \eqref{MaxCond}. Then the OCP consists of finding an admissible control $u^\ast$ such that 
   \[ V(u,x_0) =\sup\{V(u,x_0)\mid u\in  \mathbb{U}_\mathbf{B}\} \]



The elements of the OCP can be given in the following seven-tuple 
   \begin{equation}\label{OCP}
       (\mathbf{X},\mathbf{U},\mathbf{B},f,g,h,T).
   \end{equation}

\begin{assumption}\label{FilippovAssump}\rm The sets $\mathbf{X}$,  $\mathbf{U}$, and $\mathbf{B}$ are compact. The functions $f$, $g$, and $h$ are continuous.
\end{assumption}
A proof of the following theorem can be found, for instance, in Cesari \cite[Sect. 9.3.]{Cesari83} or Yong \cite[Theorem 2.2.1]{Yong2015}. 
\begin{theorem}[Filippov]\label{FilipovThm} Assume the OCP \eqref{OCP} satisfies Assumptions 1. If, for almost every $t$ in $[0,T]$, each set
        \begin{equation}\label{convexF(t,x)}
F(t,x):= \{ (\alpha, y)\in \mathbb{R}\times \mathbb{R}^n\mid  
    \alpha \leq g(t,x,u), \  y=f(t,x,u), \ u\in \mathbf{U}\},\qquad x\in X,
        \end{equation}
is convex, then there exists an optimal pair $(u^\ast,X^\ast)$.
\end{theorem} 


\todo{With ot without mathcal H, ask to David}


Define the {\it Hamiltonian}, for each  $(t,x,u,\lambda_0,\lambda)$ in $[0,T]\times \mathbf{X}\times \mathbf{U}\times\R\times\R^n$,
    \[H(t,x,u,\lambda_0,\lambda):= \lambda_0g(t,x,u) + \lambda^\top f(t,x,u).  \]  
    

\begin{assumption}\label{MPassump}\rm  %In addition to Assumptions 1, 2, and 3, the Hamiltonian satisfies the following.
\begin{enumerate}[(a)]
    \item The function $h$ is of class $\mathcal{C}^1$.
%    \item For every $(u,x,\lambda_0,\lambda)$, the function $H(\cdot,x,u,\lambda_0,\lambda)$ is measurable.
    \item For every $(t,u,\lambda_0,\lambda)$, the function $H(t,\cdot,u,\lambda_0,\lambda)$ is of class $\mathcal{C}^1$.
    \item For every $(t,x,\lambda_0,\lambda)$, the functions
     \[ H(t,x,\cdot,\lambda_0,\lambda) \mbox{ and }   H_x(t,x,\cdot,\lambda_0,\lambda)  \]
are continuous.
\end{enumerate}
\end{assumption}


The following theorem is proved in Yong \cite[Theorem 2.3.1]{Yong2015}.

\begin{theorem}[Maximum Principle] Suppose the OCP \eqref{OCP} satisfies Assumptions \ref{FilippovAssump} and \ref{MPassump} hold and the set $\mathbf{B}$ is convex. Let $(u^\ast,X^\ast)$ be an optimal pair. Then there exists a constant $\lambda_0\geq 0$ and an absolutely continuous function $\lambda:[0,T]\to\R^n$, with
    \begin{equation}\label{RegConL}
        (\lambda_0)^2 + \|\lambda(T)-\lambda_0h_x(X^\ast(T))^\top\|^2 = 1,
    \end{equation}
that satisfy
\begin{enumerate}[\rm (a)]
    \item the maximum condition, for almost every $t\in[0,T]$, 
        \begin{equation}\label{MaxCond}
             \mathcal{H}(t,X^\ast(t),u^\ast(t),\lambda_0,\lambda(t)) \geq  
             \mathcal{H}(t,X^\ast(t),u,\lambda_0,\lambda(t)) \quad \forall u\in 
             \mathbf{U},
        \end{equation}
\item the adjoint equation, for almost every $t\in[0,T]$, 
      \begin{equation}\label{AdjEq}
          \dot{\lambda}(t) = -H_x(t,X^\ast(t),u^\ast(t),\lambda_0,\lambda(t))^\top,  
      \end{equation}
\item and the transversality condition
   \begin{equation}\label{TransCond}
    [\lambda(T)^\top-\lambda_0h_x(X^\ast(T))] [y-X^\ast(T)]\geq 0 \quad \forall y\in \mathbf{B}.
   \end{equation}
\end{enumerate}
\end{theorem}


\begin{remark}\label{RemNoCons}
As pointed out by Yong \cite[p. 43]{Yong2015}, if $\mathbf{B}=\R^n$, then \eqref{TransCond} implies 
  \[\lambda(T)-\lambda_0h_x(X^\ast(T))^\top=0\]
  and so $\lambda_0=1$ by \eqref{RegConL}. In such a case, the Hamiltonian takes the form  
      \[\mathcal{H}(t,x,u,\lambda):= g(t,x,u) + \lambda^\top f(t,x,u) = H(t,x,u,1,\lambda).  \]
      Then the form of the Hamiltonian used in Section XX is justified. Further, when $h=0$, the adjoint equation \eqref{AdjEq} and the transversality condition \eqref{TransCond} become 
  %\begin{equation}
    \[    \dot{\lambda}(t) = -g_x(t,X^\ast(t),u^\ast(t))^\top -  [f_x(t,X^\ast(t),u^\ast(t))]^\top \lambda(t), \quad \lambda(T)=0
  \]
  as in \eqref{AdjEqu}.
  % \end{equation} 
  \end{remark}
 


Consider the OCP \eqref{OCP} with $\mathbf{B}=\R^n$ and $h\equiv 0$. Define
\begin{eqnarray*}
\mathcal{H}^\ast(t,x,\lambda) & := & \sup_{u\in\mathbf{U}}\mathcal{H}(t,x,u,\lambda)\\
     & = & \sup_{u\in\mathbf{U}}\{g(t,x,u) + \lambda^\top f(t,x,u)\}.    
\end{eqnarray*}

\begin{assumption}\label{ContinuityH}\rm The functions $\mathcal{H}$ and $\mathcal{H}_x$ are continuous. 
\end{assumption}


\begin{assumption}\label{piecewise}\rm The functions $u^\ast:[0,T]\to\mathbf{U}$ and $X^\ast:[0,T]\to\mathbf{X}$ satisfy the following:
\begin{enumerate}[(a)]
    \item $u^\ast$ is piecewise continuous on $[0,T]$,
    \item $X^\ast$ is continuous on $[0,T]$, 
    \item $\dot{X}^\ast$ exists and it is piecewise  continuous on $[0,T]$.
\end{enumerate}
\end{assumption}


The following theorem is proved in Seierstad and Syds\ae ter \cite[Theorem 3]{SeiSyd77}.  

\begin{theorem}\label{SufficientCond} Suppose that Assumption \ref{ContinuityH} holds. Let $(u^\ast,X^\ast)$ be an admissible pair that satisfies Assumption \ref{piecewise}. Suppose that there exists a continuous function $\lambda:[0,T]\to \R^n$ such that 
   \begin{equation}
       \mathcal{H}(t,X^\ast(t),u^\ast(t),\lambda(t)) \geq
       \mathcal{H}(t,X^\ast(t),u,\lambda(t)) \quad \forall u\in\mathbf{U},
   \end{equation}
and, except at the points of discontinuity of $u^\ast$,
    \begin{equation}
         \dot{\lambda}(t) = -\mathcal{H}_x(t,X^\ast(t),u^\ast(t),\lambda_0,\lambda(t))^\top, \quad \lambda(T)=0.
    \end{equation}
If the set $\mathbf{X}$ is convex and, for each $t$, the function $\mathcal{H}^\ast(t,\cdot,\lambda(t))$ is concave on $\mathbf{X}$, then $(u^\ast,X^\ast)$ is an optimal pair.
\end{theorem}

%\begin{assumption} The sets $\mathbf{X}$ and $\mathbf{U}$ are convex. The function $\mathcal{H}^\ast(t,\cdot,\lambda)$ is concave\end{assumption}





